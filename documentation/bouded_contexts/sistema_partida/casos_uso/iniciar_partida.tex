\subsubsection{Caso de Uso 3: Iniciar Partida}
    \textbf{Ator Principal}: Lider da Sala \\
    \textbf{Objetivo}: Iniciar a partida na sala de espera, após a configuração da partida e a entrada dos jogadores. \\
    \textbf{Pre-Requesito}: O lider da sala tem está em uma sala e todos os jogadores tem que está com estado de "pronto" \\
    \textbf{Fluxo Principal}: \\
    \begin{enumerate}
        \item O dono da sala clica no botão "Iniciar Partida".
        \item O sistema verifica se há pelo menos dois jogadores na sala.
        \item O sistema verifica se todos os participantes então em estado de "pronto"
        \item Se houver jogadores suficientes, o sistema altera o estado da sala para "Partida em Andamento".
        \item O sistema cria uma nova instância de Partida com as configurações definidas na sala.
        \item O sistema associa os jogadores da sala à instância da Partida.
        \item O sistema obtém as playlists dos jogadores participantes.
        \item O sistema inicia a primeira rodada da partida.
        \item O sistema notifica todos os jogadores da sala que a partida foi iniciada.
        \item O sistema redireciona todos os jogadores para a tela da partida.
    \end{enumerate}

    \textbf{REGRAS DE NEGOCIO}:
    \begin{itemize}
        \item \textbf{Número Mínimo de Jogadores}: Uma partida só pode ser iniciada se houver pelo menos dois jogadores na sala.
        \item \textbf{Todos os jogadores devem está prontos}: A partida só inicia quando todos os jogadores estão em estado de "pronto".
        \item \textbf{Lider da Sala Inicia}: Apenas o Lider da Sala pode iniciar a partida.
        \item \textbf{Estado da Sala}: A sala deve estar no estado "Aguardando Jogadores" para que a partida possa ser iniciada.
    \end{itemize}