\subsubsection{Caso de Uso 2: Entrar em uma Sala}
    \textbf{Ator Principal}: Jogador \\
    \textbf{Objetivo}: Entrar em uma sala de espera existente para participar de uma partida. \\
    \textbf{Pre-Requesito}: O jogador deve estar autenticado no sistema. \\
    \textbf{Fluxo Principal}: \\
    \begin{enumerate}
        \item O jogador solicita a entrada em uma sala, informando o nome ou código da sala.
        \item O sistema verifica se a sala existe.
        \item Se a sala existir, o sistema verifica se ela é pública ou privada.
        \item Se a sala for privada, o sistema solicita a senha ao jogador.
        \item O jogador informa a senha da sala (se solicitada).
        \item O sistema verifica se a senha está correta (se aplicável).
        \item O sistema verifica se a sala ainda não está cheia e se a partida ainda não foi iniciada.
        \item Se a sala estiver disponível, o sistema adiciona o jogador à sala.
        \item O sistema informa ao jogador que ele entrou na sala com sucesso.
        \item O sistema notifica os demais jogadores na sala sobre a entrada do novo jogador.
    \end{enumerate}

    \textbf{REGRAS DE NEGOCIO}:
    \begin{itemize}
        \item \textbf{Sala Cheia}: Um jogador não pode entrar em uma sala que já atingiu o número máximo de jogadores.
        \item \textbf{Partida Iniciada}: Um jogador não pode entrar em uma sala se a partida já tiver sido iniciada.
        \item \textbf{Senha da Sala}:
            \subitem Salas públicas podem ser acessadas por qualquer jogador sem senha.
            \subitem Salas privadas exigem a senha correta para acesso.
            \subitem Se a senha estiver incorreta, o sistema informa o erro ao jogador e impede a entrada na sala.
        \item \textbf{Jogador já na Sala}: Um jogador não pode entrar em uma sala na qual ele já está presente.
        \item \textbf{Estado do Jogador}:  Um jogador banido ou suspenso não pode entrar em uma sala. 
    \end{itemize}