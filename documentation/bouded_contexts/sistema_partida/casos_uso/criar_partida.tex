\subsubsection{Caso de Uso 1: Criar uma Partida}
    \textbf{Ator Principal}: Jogador \\
    \textbf{Objetivo}: Criar uma nova sala de espera para iniciar uma partida com outros jogadores, definindo as configurações da partida e a visibilidade da sala. \\
    \textbf{Pre-Requesito}: O jogador deve estar autenticado no sistema. \\
    \textbf{Fluxo Principal}: \\
    \begin{enumerate}
        \item O jogador solicita a criação de uma nova sala.
        \item O sistema solicita ao jogador que defina as configurações da partida: \\
         \subitem \textbf{Modo de Jogo}: 
            \subsubitem Casual: Sem ranking, apenas para diversão. 
            \subsubitem Ranqueado: Partida competitiva que afeta o ranking do jogador. 
            \subsubitem Battle Royale: Partida eliminatória onde o último jogador restante vence.
        \subitem \textbf{Número de Rodadas}: Campo numérico para definir a quantidade de rodadas da partida (valor mínimo: 1, valor máximo: 20).
        \subitem \textbf{Tempo Limite de Resposta}: Seletor de tempo com opções pré-definidas: (5 segundos, 10 segundos, 15 segundos, 20 segundos) ou campo numérico para definir o tempo limite para responder a cada pergunta (valor mínimo: 5 segundos, valor máximo: 30 segundos).
        \subitem \textbf{Tipo de Acerto}: \\
            \subsubitem \textbf{Música}: O jogador precisa acertar apenas o nome da música.
            \subsubitem \textbf{Artista}: O jogador precisa acertar apenas o nome do artista.
            \subsubitem \textbf{Ambos}: O jogador precisa acertar o nome da música e do artista.
        \subitem \textbf{Visibilidade da Sala}: \\
            \subsubitem \textbf{Pública}: Qualquer jogador pode entrar na sala.
            \subsubitem \textbf{Privada}: Apenas jogadores convidados ou com a senha de acesso podem entrar na sala.
        \item O jogador define as configurações desejadas.
        \item Se o jogador escolher "Privada", o sistema solicita que ele defina uma senha de 4 dígitos para a sala.
        \item O jogador define a senha da sala, caso não pula esse passo.
        \item O sistema valida as configurações.
        \item Se as configurações forem válidas, o sistema cria uma nova sala com as configurações definidas, o jogador como dono e a senha caso seja privada.
        \item O sistema adiciona o jogador à sala.
        \item O sistema informa ao jogador que a sala foi criada com sucesso, incluindo o nome da sala e a senha.
    \end{enumerate}

    \textbf{REGRAS DE NEGOCIO}:
    \begin{itemize}
        \item \textbf{Sala Cheia}: Um jogador não pode entrar em uma sala que já atingiu o número máximo de jogadores definido na criação da sala.
        \item \textbf{Partida Iniciada}: Um jogador não pode entrar em uma sala se a partida já tiver sido iniciada.
        \item \textbf{Senha da Sala}
            \subitem A senha deve ter exatamente 4 dígitos sendo todos numericos.
            \subitem Para salas públicas, qualquer jogador pode entrar sem necessidade de senha.
            \subitem Para salas privadas, o jogador precisa fornecer a senha correta para entrar.
            \subitem Se o jogador fornecer uma senha incorreta, o sistema deve exibir uma mensagem de erro e impedir a entrada na sala.
        \item \textbf{Jogador já na Sala}:  Um jogador não pode entrar em uma sala na qual ele já está presente.
        \item \textbf{Estado do Jogador}: 
            \subitem Um jogador não pode entrar em uma sala se estiver banido ou suspenso.
            \subitem Um jogador que foi desconectado de uma partida em andamento pode reentrar na mesma sala, caso a partida ainda esteja acontecendo.
            \subitem O criador da sala é classificado como dono da sala
            \subitem O dono da sala pode definir configurações adicionais, como expulsar jogadores, iniciar a partida e alterar as configurações da sala.
            \subitem O dono da sala sair da sala, a liderança é trasferida para outro jogador presente na sala.
            \subitem O dono da sala pode trasferir a liderança para outro jogador presente na sala.
        \item \textbf{Disponibilidade da Sala}: 
            \subitem Uma sala pode ser removida da lista de salas disponíveis se a partida for iniciada ou se o dono da sala a cancelar.
            \subitem Uma sala privada pode ser removida da lista de salas disponíveis se o dono da sala a fechar ou se tornar inacessível após um determinado tempo de inatividade.
        \item \textbf{Conexão com a Sala}:
            \subitem O sistema deve garantir que o jogador tenha uma conexão estável com a sala.
            \subitem Se o jogador perder a conexão com a sala, ele deve ser removido da lista de jogadores e notificado.
            \subitem O sistema deve tentar reconectar o jogador automaticamente, caso a conexão seja perdida temporariamente.     
    \end{itemize}